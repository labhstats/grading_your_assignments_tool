% !TEX program = pdflatex
\documentclass[12pt,a4paper]{article}

\usepackage[utf8]{inputenc}
\usepackage[T1]{fontenc}
\usepackage[norsk]{babel}
\usepackage[a4paper,margin=25mm]{geometry}
\usepackage{lmodern}
\usepackage{microtype}
\usepackage{xcolor}
\usepackage{fancyhdr}
\usepackage{lastpage}

% Samme metadata som i besvarelses-PDFen (fylles ut av lærer)
\newcommand{\CourseCode}{\textbf{KURS-101}}
\newcommand{\CourseName}{\textbf{Kursnavn}}
\newcommand{\AssignmentName}{\textbf{Oppgavetekst}}
% Innleveringsfrist (settes av lærer)
\newcommand{\DueDate}{\textbf{06.07.2067 kl: 13:37}}
\newcommand{\DocType}{\textbf{Oblig X}}
\newcommand{\DocVariant}{\textbf{Vår 2026}}

\pagestyle{fancy}
\fancyhf{}
\fancyhead[L]{\CourseCode{} -- \CourseName}
\fancyhead[R]{\DocType{}\;|\;\DocVariant}
\fancyfoot[L]{\AssignmentName}
\fancyfoot[C]{Side \thepage\ av \pageref{LastPage}}
\fancyfoot[R]{}
\renewcommand{\headrulewidth}{0.4pt}
\renewcommand{\footrulewidth}{0.4pt}

\setlength{\parindent}{0pt}
\setlength{\parskip}{6pt}

\newcommand{\QuestionText}[2]{%
  % Hold hele oppgaven samlet på én side (forutsetter at teksten er kort nok til å få plass på én side)
  \par\noindent\begin{minipage}{\linewidth}
  {\Large \textbf{Oppgave #1}}\\[2mm]
  \textcolor{gray}{\rule{\linewidth}{0.4pt}}\\[3mm]
  #2
  \end{minipage}
  \par\vspace{6mm}
}

\begin{document}

% Forside (valgfritt)
\begin{center}
{\LARGE \AssignmentName\par}
\vspace{2mm}
{\Large \CourseCode{} -- \CourseName\par}
\vspace{2mm}
\begin{tabular*}{0.9\linewidth}{@{\extracolsep{\fill}}l c r@{}}
\DocType & \textbf{Frist:} \DueDate & \DocVariant\\
\end{tabular*}
\end{center}

% Spørsmålene flyter fortløpende og bruker så få sider som nødvendig.
% Bruk: \QuestionText{<nr>}{<oppgavetekst>}
\QuestionText{1}{(sett inn oppgave 1 her)}
\QuestionText{2}{(sett inn oppgave 2 her)}
\QuestionText{3}{(sett inn oppgave 3 her)}
\QuestionText{4}{(sett inn oppgave 4 her)}
\QuestionText{5}{(sett inn oppgave 5 her)}

\end{document}